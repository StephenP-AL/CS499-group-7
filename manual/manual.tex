\documentclass{article}
\setlength\parindent{0pt}
\setlength{\parskip}{0.85\baselineskip}
\title{\textbf{Truckr}\\ \large{User Manual}}
\date{\vspace{-5ex}}
\usepackage[margin=1in]{geometry}
\usepackage[hidelinks]{hyperref}
\usepackage{xcolor}
\usepackage{graphicx}
\graphicspath{{img/}}
\newcommand{\imgbox}[2]{\fcolorbox{darkgray}{white}{\includegraphics[width=#1px]{#2}}}
\newcommand{\cont}[1]{\noindent \fcolorbox{darkgray}{orange}{\begin{minipage}{300px} \section{#1} \end{minipage}}\\}
\newcommand{\scont}[1]{\noindent \fcolorbox{darkgray}{olive}{\begin{minipage}{300px} \subsection{#1} \end{minipage}}\\}
\newcommand{\degree}[1]{{#1}$^{\circ}$}

\renewcommand{\familydefault}{\sfdefault}

\usepackage{hyperref}
\begin{document}
\begin{center}
\includegraphics[width=300px]{TruckrLogo.jpg}\\
	\Huge{User Manual}
\end{center}
\begin{center}
\end{center}


\noindent \fcolorbox{darkgray}{lightgray}{\begin{minipage}{300px} \tableofcontents \end{minipage}}\\\\\\

	\newpage
\cont{Installation}
\begin{itemize}
	\item Install python 3 using your operating system's package manager, or from https://www.python.org.\\
		For Debian based Linux distributions (such as Ubuntu):
		\begin{verbatim}
		sudo apt install python3
		\end{verbatim}
		
	\item Install Django from the command line. 
		\begin{verbatim}
		pip install django
		\end{verbatim}

	\item Extract the Truckr application to a local drive.

\end{itemize}

\cont{Start Truckr}

\begin{itemize}
	\item From your command line, navigate to the subdirectory titled 'seven'.
	\item Execute the following command:
		\begin{verbatim}
		python3 manage.py runserver
		\end{verbatim}
		This will start and run the testing server.
	\item In your web browser, navigate to 127.0.0.1:8000
	\item To stop the server, press Ctrl + C in your terminal.
\end{itemize}

\cont{Log In}

From the home page, click Sign In.

\imgbox{300}{home.PNG}

This will bring you to the sign-in page.

\imgbox{300}{signin.PNG}

Use the following accounts to test the features of the four different account types.

\begin{tabular}{| l | l | l |}
	\hline
	Account Type & username & password \\
	\hline
	Driver & squir & groupseven \\
	Maintenance & moosb & groupseven \\
	Shipping Manager & fatan & groupseven \\
	Full Account & badeb & groupseven \\
	\hline
\end{tabular}

Once signed in, you will see a welcome screen, and a navigation bar.

\imgbox{300}{signed.PNG}

To sign out, click Home on the navigation bar. \\

On the Home screen, click the Sign Out button.

\cont{Driver Functions}

Driver accounts can view their own incoming and outgoing shipments.

\scont{Incoming Shipments}

Click Incoming Shipments on the navigation bar.

The incoming shipments page will display.

\imgbox{500}{drivershipin.PNG}

You can view the purchase order associated with each shipment by clicking the View button in the Purchase Order field.

\imgbox{300}{purchaseorderdetail.PNG}

Note that driver accounts may not add or edit items in the purchase order.

\scont{Outgoing Shipments}

Click Outgoing Shipment on the navigation bar.

The outgoing shipments page will display.

\imgbox{500}{drivershipout.PNG}

You can view the purchase order as described in the previous section.

You can also view the manifest associated with the shipment by clicking View in the Manifest field.

\imgbox{300}{manifestdetail.PNG}

\cont{Shipping Manager Functions}

Shipping managers can create, view, and edit all incoming and outgoing shipments.

\scont{Create New Shipment}

Click Incoming Shipment or Outgoing Shipment on the navigation bar. 

This will bring up the corresponding shipment page.

\imgbox{300}{manageshipin.PNG}

At the bottom of the screen, click the Add Incoming Shipment or Add Outgoing Shipment button.

This will bring up the appropriate form to create a new shipment.

\imgbox{300}{addship.PNG}

Complete the form and click Save at the bottom of the screen.

Note: if you need to create a new purchase order for your shipment, you can click Add next to Purchase orders on the left side of the screen.

\imgbox{200}{shipadminmenu.PNG}

\scont{Edit Shipment}

Click Incoming Shipment or Outgoing Shipment on the navigation bar. 

This will bring up the corresponding shipment page.

\imgbox{300}{manageshipin.PNG}

Locate the shipment you wish to edit, and click the corresponding Edit button in the Edit field.

This will bring up a form to edit the shipment.

\imgbox{200}{editshipment.PNG}

After you have made your changes, click Save at the bottom of the page.

\scont{Shipping Reports}

Click Incoming Shipment or Outgoing Shipment on the navigation bar. 

This will bring up the corresponding shipment page.

\imgbox{300}{manageshipin.PNG}

Click the Monthly Shipping Reports button. This will bring up a listing of reports up to the current month.

\imgbox{300}{shipreportlist.PNG}

Click Select Report to view the report.

\imgbox{300}{shipreport.PNG}

Note: The demonstrations data contains shipments dated 2023/02, 2022/02, and 2022/03. All other reports will be empty.

\cont{Maintenance Functions}

Maintenance accounts can view, edit, and create all vehicle and maintenance records.

\scont{Add a Vehicle}

From the list of vehicles, click the Add Vehicle button at the bottom of the page to bring up the Add vehicle form.

\imgbox{200}{addvehicle.PNG}

If the vehicle requires a new parts list, click Add next to Parts lists on the menu on the left side of the page.

\imgbox{150}{maintadminmenu.PNG}

\scont{View Vehicle Information}

Click Vehicles in the navigation bar to bring up a list of vehicles.

\imgbox{300}{vehiclelist.PNG}

Click the Details button in the Details field to view more information about the vehicle

\imgbox{300}{vehicledetail.PNG}

\scont{Vehicle Report}

From the vehicle detail page, click the Vehicle Maintenance Report button.
\end{document}
